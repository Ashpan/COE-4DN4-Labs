\documentclass[titlepage]{article}
\usepackage[utf8]{inputenc}
\usepackage[english]{babel}
\usepackage[margin=0.7in]{geometry}
\usepackage{amssymb}
\usepackage{amsmath}
\usepackage{mathrsfs}
\usepackage{graphicx}
\usepackage{float}
\usepackage{enumitem}
\usepackage{listings}

\graphicspath{ {Images/} }
\hyphenpenalty=10000
\renewcommand{\familydefault}{\sfdefault}

\lstdefinelanguage{pseudocode}{
    morekeywords={if, else, then, print, end, for, do, while, Let},
    morecomment=[l]\#
}
\lstset{
    basicstyle=\ttfamily,
    keywordstyle=\bfseries,
    showstringspaces=false,
    tabsize=4,
    frame=top,
    mathescape=true,
    frame=bottom,
    numbers=left,
    commentstyle=\color{gray},
}
\title{COMPENG 4DN4: Laboratory 2\\Online Grade Retrieval Application}
\author{Ashpan Raskar - raskara - 400185326\\Dharak Verma - vermad1 - 400114166}
\date{}

\begin{document}
    \maketitle
    \newpage

    \begin{section}{Member Contributions}
        \begin{enumerate}
            \item Ashpan Raskar \\
            Created the server class and its class fields and methods. Created the main method.
            \item Dharak Verma \\
            Created the client class and its class fields and methods.
3
        \end{enumerate}

    \end{section}

    \begin{section}{Server Pseudo Code}
        \begin{verbatim}
            Setup the host based on the current IP address
            Configure the port and the file name from the .ENV file
            Create a dictionary of commands to easily reference data from the CSV
            Bind the socket to the host and port
            Run the print course grades function
            Run the main server program function

            define print course grades function
                Open the CSV file
                Read the CSV file
                Clean the CSV file
                Print the course grades
                Parse the data into a class field dictionary
                Close the CSV file

            define clean data function
                Remove the newline character from the data
                Remove the double quotes from the data
                Remove the spaces from the data
                Remove the empty strings from the data
                Return the cleaned data

            define parse data function
                Create a dictionary to store the data
                Iterate over the data
                    Split the data into a list
                    Store the data in the dictionary
                Return the dictionary

            define the run command function
                if the command is "GG"
                    append all the students grades to a string
                if the command is "GMA"
                    append the students average midterm grade to a string
                if the command is "GEA"
                    append the students average exam grade to a string
                if the command starts with "GL"
                    append the students lab grade based on the command to a string using the dictionary
                return the string

            define main server program function
                Listen for connections
                Accept the connection
                while true:
                    Receive the data
                    Run the run command function
                    Encrypt the data using their secret key from the run command function
                    Send the data
                Close the connection
        \end{verbatim}
    \end{section}
    \begin{section}{Client Pseudo Code}
        \begin{verbatim}
            Setup the host based on the current IP address
            Configure the port from the .ENV file
            Create a dictionary of student numbers and their secret keys for testing purposes
            Create a dictionary of commands to create user readable prints
            Create a socket to connect to the server with the host and port
            Run the main client program function

            define the main client program function
                while the command isn't "bye"
                    Take in the student number and the command in from the user
                    Print the command being run using the dictionary to the user
                    Send the student number and the command to the server
                    Receive the data from the server
                    Decrypt the data using the secret key from the dictionary
                    Print the data to the user
        \end{verbatim}
    \end{section}
    \begin{section}{Explanation}
        The general workflow of the program is as follows.
        \begin{enumerate}
            \item The client sends their student number and the command they want to run to the server.
            \item The server receives the command and runs the appropriate function to compute the value get the data.
            \item This data is then encrypted using the secret key of the student.
            \item The data is then sent back to the client.
            \item The client receives the data and decrypts it using the secret key.
            \item The client prints out the data to the user.
        \end{enumerate}
        To begin the program, the server reads the data from the CSV file and stores it in a dictionary for ease of usage. The server prints all the data, and then binds the socket to the host and port. The server then runs the main server program function. Meanwhile the client also binds the socket to the host and port and is waiting for user input.
        The parse data function works by creating a dictionary of data in the following format:
        \begin{verbatim}
            {
                student_number_1: {
                    "key": secret_key,
                    "Student Name": student_name,
                    "Midterm": midterm_grade,
                    "Lab 1": lab_1_grade,
                    "Lab 2": lab_2_grade,
                    "Lab 3": lab_3_grade,
                    "Lab 4": lab_4_grade,
                    "Exam 1": exam_1_grade,
                    "Exam 2": exam_2_grade,
                    "Exam 3": exam_3_grade,
                    "Exam 4": exam_4_grade
                },
                student_number_2: {
                    "key": secret_key,
                    "Student Name": student_name,
                    "Midterm": midterm_grade,
                    "Lab 1": lab_1_grade,
                    "Lab 2": lab_2_grade,
                    "Lab 3": lab_3_grade,
                    "Lab 4": lab_4_grade,
                    "Exam 1": exam_1_grade,
                    "Exam 2": exam_2_grade,
                    "Exam 3": exam_3_grade,
                    "Exam 4": exam_4_grade
                }, ...
            }
        \end{verbatim}
        In the run command function, for the "GLx" commands, the function checks if the command starts with the string "GL" and if it does, it uses the dictionary to check the command with the field in the dictionary to fetch the data. For example if they enter "GL3", then the dictionary value for the key "GL3" will be "Lab 3" and for each student number it will fetch the grade for "Lab 3" and finally take the average of all the grades and return it.

        In the encrypt and decrypt string functions, the fernet encryption method is used with a symmetric secret key (the same key for encryption and decryption), this is encoded/decoded using UTF-8 character set.

        The main method works by initializing the server or the client based on the flag being passed when running the file using the "-r" or "--role" flag. If it is run with the "-r client" flag, then the client class object is initialized, if it is run with the "-r server" flag, then the server class object is initialized.
    \end{section}
\end{document}